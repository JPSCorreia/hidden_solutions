\section{Introdução} \label{section: Introducao}
\setstretch{1.5}
\par \vspace{8pt}
O presente trabalho tem como objetivo a criação de uma empresa, abordando os aspetos fundamentais necessários à sua constituição e organização. O trabalho será desenvolvido por um grupo de três formandos, que colaborará na elaboração de um documento que sintetize a conceção e estrutura da empresa.
\par \vspace{10pt}
Ao longo deste trabalho, serão apresentados o nome e logotipo da empresa, a sua finalidade, dimensão, setor de atividade, capital social, forma jurídica, missão, visão e valores. Também será abordada a estrutura organizacional, incluindo os recursos humanos, organograma, funções principais e tipo de comunicação.
\par \vspace{10pt}
Além disso, será feita uma análise detalhada das metas empresariais, com a definição de objetivos de curto, médio e longo prazo. Estes objetivos servirão como guia estratégico para o crescimento da empresa, alinhados com a sua visão e valores. Esta abordagem permitirá a compreensão das necessidades operacionais e das estratégias para alcançar um crescimento sustentável.
\par \vspace{10pt}
Por último, o trabalho culmina com a apresentação de um plano de comunicação organizacional que garantirá a coesão e eficiência da empresa. Este plano incluirá a estruturação dos canais de comunicação internos e externos, assegurando que todos os membros da equipa estão alinhados com os objetivos e a cultura da organização.
\par \vspace{10pt}
O trabalho está organizado em três partes principais: Introdução, Desenvolvimento e Conclusão. Na Introdução, são expostos os objetivos do trabalho e a sua estrutura geral. O Desenvolvimento focará a construção detalhada da empresa, seguindo a estrutura indicativa fornecida. Por fim, na Conclusão, serão sintetizados os pontos essenciais apresentados ao longo do trabalho.
\par \vspace{10pt}
Com este trabalho, procuramos aplicar os conhecimentos adquiridos no âmbito da UFCD 5065 - Empresa: Estrutura e Funções, promovendo uma reflexão crítica e criativa sobre a constituição e gestão de uma empresa.